Es gibt zwei verschiedne Arten von Widgets in Flutter: Stateless und Stateful Widgets. Stateful Widgets sind Widgets, die sich ver�ndern k�nnen, wenn z.B. Daten dargestellt werden oder der Benutzer mit dem Widget interagieren kann. Stateless Widgets hingegen sind Widgets, die sich nie ver�ndern\cite{reg407}.\\
State ist Information, die ermittelt werden kann, w�hrend ein Widget gebaut wird und sich auch �ndern kann. Beim Entwickeln einer App in Flutter muss also darauf geachtet werden, dass beim Auftreten solcher �nderungen ein Widget auch neu aufgebaut wird, da die dahinterliegenden Informationen sich sonst ge�ndert haben, aber nicht das Widget selbst. Das kann mit Methoden wie z.B. setState geregelt werden\cite{reg408}. Hier k�nnen aber Probleme entstehen, wenn State �ber verschiedene Klassen hinweg weitergegeben werden muss; Dies kann zwar mit �bergabeparametern gel�st werden, aber auch hier bleibt das genannte Problem bestehen und vor allem bei gr��eren Projekten ist die L�sung bei weitem nicht ideal.\\
Im Internet k�nnen viele Klassen gefunden werden, die State Management vereinfachen sollen. Eine der Basisklassen ist die InheritedWidget class. Die Logik hinter der Klasse ist, dass State im Baum nach unten vererbt werden kann, sich die verkn�pften Widgets sich bei einer �nderung des States in einem dar�berliegenden Widget also auch neu aufbauen\cite{reg409}. 
F�r das Projekt Sleepanalyzer wurde die Klasse provider zusammen mit ChangeNotifier verwendet, was einer der popul�rsten Ans�tze ist. Klassen k�nnen als ChangeNotifier Klasse angelegt werden. Andere Klassen k�nnen dann subscriben, um bei �nderungen verst�ndigt zu werden. Die Klasse Provider sorgt daf�r, dass State nicht durch den ganzen Widgetbaum weitergegeben werden muss, sondern direkt zu einem bestimmten Widget weitergegeben werde kann. Dazu muss allerdings eine Liste in einem Widget �ber einem Provider und dem Widget, in dem der State gebraucht wird angelegt sein\cite{410}.