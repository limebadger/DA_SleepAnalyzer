\fancyfoot[RE,LO]{Autor: Sumaiya Hossain}

\section {Platine} \label {}

\subsection {RTC-Modul} \label {}

In diesem Kapitel wird das RTC Modul pr�sentiert, welches f�r das Projekt unverzichtbar ist. 

\subsubsection {Hintergrund} \label {}

Da der Benutzer eine bestimmte Uhrzeit einstellen kann f�r den Wecker, m�ssen mit den Messungen auch die Uhrzeit mitgegeben werden. Damit das m�glich ist wurde ein RTC-Modul gefertigt. 

\subsubsection {Bestandteile} \label {}

Die Schaltung des Real-Time-Clocks besteht aus einem Quarz, einer Knopfzellenbatterie und dem DS1302SN+. Das DS1302SN+ ist ein Trickle-Charge Timekeeping Chip, die auch ein Echtzeituhr und -kalender beinhaltet und sie sorgt daf�r, dass die App beim Auswerten der Messungen die genau Zeit kennt. Der Unterschied zum gew�hlten Chip und zum DS1302 ist, dass der gew�hlte eine andere Pin Package hat und die Temperatur Range geht von -40�C bis 85�C. Die Batterie dient dazu die Echtzeituhr am laufenden zu halten, wenn das RTC nicht versorgt wird.  

\subsubsection {Funktion} \label {}

Durch den Quarz wird eine Taktfrequenz erzeugt, welches erm�glicht, dass den Sekundentakt einer Uhr zu generieren.

\subsubsection {EOG-Platine} \label {}

Ein wichtiger Bestandteil dieses Projekts ist das EOG (siehe Kapitel Claudia), f�r die eine Platine angefertigt wurde. 
Die meisten SMD Bauteile waren verf�gbar sowohl in der Schule als auch in den Gesch�ften. Ein Problem gab es bei den 3M Ohm SMD Widerst�nde, die dann mit drei seriell geschalteten 1M Ohm SMD Widerst�nden ersetzt wurden. 
Die Platine wurde auf KiCad Version 6.0, ein Schaltplaneditor und Layoutprogram, angefertigt. Zuerst wurde eine Schematic erstellt, in der alle Bauteile verbunden wurden. Danach folgte die Footprint-Zuweisung der Bauteile. Ein Problem gab es bei dem ESP, denn eine passende Footprint war nicht vorhanden und musste daher erstellt werden. Anschlie�end konnte dann nach der Erstellung der PCB die eigentlichen Leiterbahnen f�r die Platinen nachgezogen werden. 
